\chapter{Défis avec les benchmarks}

Le premier défi avec les benchmarks était qu'on avait peu d'expérience avec ces outils. Le problème avec cela est que nous n'étions pas sûrs de savoir quelles métriques sont les plus significatives et utiles pour les comparaisons ainsi que comment bien utiliser les paramètres pour avoir des benchmarks valides. La solution pour ce défi est de tout simplement lire leur documentation ainsi que de réfléchir sur les métriques importantes pour la mise en situation. \newline

Le deuxième défi rencontré est que pour le test du disque, où il faut mentionner le nom du disque à tester, les instances AWS et Azure n'ont pas le même nom et il a fallu quelques essais pour être en mesure de faire le script de benchmark pour outre-passer cette différence. \newline

Nous avons également eu quelques difficultés à obtenir certaines caractéristiques des machines. Comme on peut le voir dans la partie \ref{machines}, certaines informations sont manquantes (par exemple, il n'y a pas le modèle de carte réseau pour toutes les instances). La virtualisation ne nous permet pas de résoudre ce problème. 
